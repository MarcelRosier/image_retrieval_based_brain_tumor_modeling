% !TeX root = ../main.tex
% Add the above to each chapter to make compiling the PDF easier in some editors.

\chapter{Introduction}\label{chapter:introduction}

Magnetic resonance imaging (MRI) is the state of the art diagnostic tool to diagnose and locate brain tumors. While being able to reliably highlight areas of sufficiently high tumor cell concentration, it struggles with areas of lower density.
For glioblastoma (GBM), which is one of the most aggressive types of brain tumor that characteristically infiltrates surrounding tissue, the shortcomings of MRI scans often lead to tumor recurrence after treatment.\\
Current treatment plans try to account for the unknown infiltration by also uniformly targeting the visible tumor's surrounding.
While decreasing the probability of secondary tumors, this has the significant drawback of typically still not resecting all infiltrations and also often unnecessarily damaging healthy tissue, which has a negative impact on the patient's life quality.
Personalizing the target of radio therapy by complementing the MRI scans with individual simulations that give information about the spatial distribution of tumor cell concentration could preserve healthy tissue and reduce the likelihood of residual tumor cells \parencite{Stupp2014} \parencite{Lipkova2019} \parencite{AdultGlioAlexander}.\\
Conventional approaches to implement a personalization try to simulate the tumor growth for each individual patient based on differential equations. 
However, utilizing highly efficient numerical solvers still results in extreme runtimes that obstruct transfer into clinical practice
\parencite{ezhov2021geometryaware}.
To overcome this issue \parencite{LearnMorphInfer} introduced the learn-morph-infer pipeline that delivers fairly accurate results in near real-time.
Since it relies on a neural network to predict patient-specific parameters, it is nondeterministic and might not be robust to all inputs.
In the following, we propose an image retrieval based approach that performs a query of a patient specific scan to a database of synthetic tumors, returning an ideally close match of the patient's tumor. This can function as a replacement for the predicting network and forward solver within the pipeline.
As a baseline, this image retrieval process is achieved via a primitive iterative pair wise comparison.
We investigate how down sampling as well as compression using autoencoders and variational autoencoders can improve runtimes, while assessing the quality of results by comparison with the iterative baseline.\\
\todo{results and point to future work}







